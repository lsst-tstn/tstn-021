\documentclass[TS,authoryear,toc]{lsstdoc}
% lsstdoc documentation: https://lsst-texmf.lsst.io/lsstdoc.html
\input{meta}

% Package imports go here.

% Local commands go here.

%If you want glossaries
%\input{aglossary.tex}
%\makeglossaries

\title{Observatory User-Documentation Working Group Charge}

% Optional subtitle
% \setDocSubtitle{A subtitle}

\author{%
Leanne Guy
}

\setDocRef{TSTN-021}
\setDocUpstreamLocation{\url{https://github.com/lsst-tstn/tstn-021}}

\date{\vcsDate}

% Optional: name of the document's curator
% \setDocCurator{The Curator of this Document}

\setDocAbstract{%
A key component in the transitions from construction, commissioning and through to operations is the delivery of a suite of cross-subsystem user-level documentation that will be used by both commissioning and operations-era observatory staff. With the rapid commissioning of systems and early observing activities now underway, capturing high-level system usage information, lessons learnt, and having an area to host reference content while performing integration activities is essential. 
}

% Change history defined here.
% Order: oldest first.
% Fields: VERSION, DATE, DESCRIPTION, OWNER NAME.
% See LPM-51 for version number policy.
\setDocChangeRecord{%
  \addtohist{1}{YYYY-MM-DD}{Unreleased.}{Leanne Guy}
}


\begin{document}

% Create the title page.
\maketitle
% Frequently for a technote we do not want a title page  uncomment this to remove the title page and changelog.
% use \mkshorttitle to remove the extra pages

% ADD CONTENT HERE
% You can also use the \input command to include several content files.

\appendix
% Include all the relevant bib files.
% https://lsst-texmf.lsst.io/lsstdoc.html#bibliographies
\section{References} \label{sec:bib}
\renewcommand{\refname}{} % Suppress default Bibliography section
\bibliography{local,lsst,lsst-dm,refs_ads,refs,books}

% Make sure lsst-texmf/bin/generateAcronyms.py is in your path
\section{Acronyms} \label{sec:acronyms}
\addtocounter{table}{-1}
\begin{longtable}{p{0.145\textwidth}p{0.8\textwidth}}\hline
\textbf{Acronym} & \textbf{Description}  \\\hline

AD & Associate Director \\\hline
CSC & Commandable SAL Component \\\hline
DM & Data Management \\\hline
T\&S & Telescope and Site \\\hline
TS & Test Specification \\\hline
TSTN & Telescope and Site Technical Note \\\hline
\end{longtable}

% If you want glossary uncomment below -- comment out the two lines above
%\printglossaries





\end{document}
